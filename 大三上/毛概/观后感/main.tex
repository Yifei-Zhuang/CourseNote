\documentclass[utf8,a4paper,20pt]{article}

% Language setting
% Replace `english' with e.g. `spanish' to change the document language
% \usepackage[UTF8]{ctex}
\usepackage{xeCJK}
\setCJKmainfont{Songti TC Light} %衬线字体 缺省中文字体为
% \setCJKmainfont{Lantinghei SC} %衬线字体 缺省中文字体为
\setCJKsansfont{Heiti SC} %serif是有衬线字体sans serif无衬线字体。
\setCJKmonofont{FangSong} %中文等宽字体
% Set page size and margins
% Replace `letterpaper' with `a4paper' for UK/EU standard size
\usepackage[letterpaper,top=2cm,bottom=2cm,left=3cm,right=3cm,marginparwidth=1.75cm]{geometry}
\usepackage{placeins}
% Useful packages
\usepackage{amsmath} 
\usepackage{graphicx}
\usepackage{setspace}
\renewcommand{\baselinestretch}{1.5}
\usepackage[colorlinks=true, allcolors=blue]{hyperref}
% 页眉页脚
\usepackage{fancyhdr}
\pagestyle{fancy}
\lhead{}
\rhead{}
\chead{}
\chead{《零容忍》纪录片观后感}
\cfoot{\thepage}
\renewcommand{\headrulewidth}{0.4pt}
\renewcommand{\footrulewidth}{0.4pt}
\setlength{\parskip}{0.5em}
\title{{\huge {《零容忍》纪录片观后感}}}
\author{{\Large{软工2002 3200105872 庄毅非}}}
\date{}
\begin{document}

\maketitle
\large
\par 
    为一个比较关注时政的人,在此前我就曾经看过《零容忍》的全5集纪录片,从中有了一定的收获,这次我也希望能够借书写观后感的机会分享一下自己的感想。
\par 
    正如党的二十大报告所指出的:“只要存在腐败问题产生的土壤和条件,反腐败斗争就一刻不能停,必须永远吹冲锋号”。从建国以来,大力惩治贪腐官员就是我国
    从严治党理念的一个重要举措,早先有建国时期刘青山张子善因贪污地方款项被毛主席命令执行枪决,后有赖小民贪污巨额国有资产被判处死刑,我党向来对贪腐
    行为重拳出击,绝不姑息。自十八大以来,我党坚持全面从严治党,强力打击党内的各种违纪违法行为,强烈的历史责任感、深沉的使命忧患感、顽强的意志品质
    ,深入推进党风廉政建设和反腐败斗争,坚持无禁区、全覆盖、零容忍,严肃查处腐败,坚决遏制腐败蔓延势头,形成了反腐斗争压倒性态势。经过10年的整治,
    我党的精神风貌也焕然一新。这部纪录
    片就是列出党自18大以来打击的贪腐官员的典型,表现党坚持自我革命、坚持全面从严治党的战略方针、一刻不停推进党风廉政建设和反腐斗争过程,杀一儆百,
    警示官员们要正风守纪,也是对包括我在内的广大人民表现党中央强力惩治贪腐行为的决心。
\par 
    在全片中,让我印象最深刻的就是第三集《惩前毖后》中的原北京市规划委员会书记、副市长、市委常委的陈刚的故事。他曾经就读于清华大学建筑学院,在校园中钻研城市建设课题。毕
    业后加入北京市城市建设规划机关,仅仅8年就成为机关的骨干人员。他本来可以在改革开放的时代背景下,在国家经济蓬勃发展、城市建设日新月异的时代大背
    景下发挥自己所学,利用自己的才华建设国家的首都。这是多么让人羡慕的机遇,但是他却没有在利益面前坚持自己作为党员的底线,收受贿赂,数额和形式越来
    越触目惊心,最后落入违纪违法的深渊中无法自拔。他最后在纪录片中也反思说,他自己主持建设的那些私家园林和能影响大多数市民的首都城建规划事业相比,
    根本是不值一提的,但是后悔为时已晚。它的故事对我也是一种警醒,作为同样在国家高校就读的我们来说,我们也要时刻警醒自己未来步入社会参加工作之后也
    绝不能干违法违纪的事情,避免自己的一腔理想和才华施展到错误的地方。
\par 
    正如习主席所说,江山就是人民,人民就是江山。中国共产党领导人民打江山、守江山,守的是人民的心。“不得罪成百上千的腐败分子,就要得罪十四亿人民,这
    是一笔再明白不过的政治账、人心相背帐”。《零容忍》中的各种案例,都表现了反腐败斗争的艰巨性和长期性。保持党的纯洁性和先进性不是一件容易的事,但是
    这是党作为人民先锋队的前提之一。最后,我希望以二十大报告的一句话作结:“必须持之以恒推进全面从严治党,深入推进新时代党的建设新的伟大工程,以党的
    自我革命引领社会革命“。党只有永葆自我精神,坚持从严治党,大力打击违纪违法行为,坚持敢于直面病灶、刮骨疗毒的决心,才能够赢得更广大人民的拥护。
    

\end{document}